%%
%% GENERIC ENGLISH INVOICE
%%
\documentclass[english]{article}
\usepackage[T1]{fontenc}
\usepackage[utf8]{inputenc}
\usepackage{geometry}
\geometry{verbose,letterpaper,tmargin=1cm,bmargin=1cm,lmargin=1cm,rmargin=1cm}
\usepackage{array}
\usepackage{graphicx}
\usepackage{setspace}
\usepackage{longtable}
\fontfamily{cmss}\fontsize{10pt}{12pt}\selectfont
\renewcommand{\familydefault}{\sfdefault}
\usepackage{helvet}
\makeatletter


%%%%%%%%%%%%%%%%%%%%%%%%%%%%%% LyX specific LaTeX commands.
%% Because html converters don't know tabularnewline
\providecommand{\tabularnewline}{\\}

\makeatother

\usepackage{babel}

\begin{document}
\noindent \includegraphics{(company_logo)}
		

\noindent \begin{tabular}{>{\centering}p{0.5\columnwidth}>{\centering}p{0.5\columnwidth}}
\noindent \begin{flushleft}
\textbf{To:}

(customer\_contact)

(customer\_name)


(customer\_address1\_street)

(customer\_address1\_city)

(customer\_address1\_state)

(customer\_address1\_country)

(customer\_address1\_zipcode)

\par\end{flushleft}
& \begin{flushleft}
\textbf{From:}

(company\_name)

(company\_address1\_street)

(company\_address1\_city)

(company\_address1\_state)

(company\_address1\_country)

(company\_address1\_zipcode)

\vspace{5mm}

\begin{tabular}{ll}
\textbf{Email} & (company\_contact\_email) \tabularnewline
\textbf{Phone} & (company\_contact\_phone) \tabularnewline
%% if company\_contact\_fax
%% \textbf{Fax} & (company\_contact\_fax) \tabularnewline
%% end
\end{tabular}

\par\end{flushleft}
\tabularnewline
\end{tabular}

\noindent \vspace{10mm}


\noindent \textbf{TAX INVOICE}

\vspace{5mm}
\noindent \begin{tabular}{ll}
%% foreach taxes
%% (name_tax) & (taxnumber) \tabularnewline
%% end
\end{tabular}

\noindent \vspace{10mm}


\noindent \begin{tabular}{ll}
\textbf{Account ID} & (code\_customer) \tabularnewline
\textbf{Invoice No} & (code\_invoice) \tabularnewline
\textbf{Order No} & (code\_ordernumber) \tabularnewline
\textbf{Date Created} & (date\_trans) \tabularnewline
\textbf{Date Due} & (date\_due) \tabularnewline
\end{tabular}

\noindent \vspace{10mm}

\noindent \begin{tabular}{>{\raggedright}p{0.2\columnwidth}>{\raggedright}p{0.3\columnwidth}>{\raggedright}p{0.1\columnwidth}>{\raggedright}p{0.1\columnwidth}>{\raggedright}p{0.1\columnwidth}>{\raggedright}p{0.1\columnwidth}}
\textbf{Number} & \textbf{Description} & \textbf{Quantity} & \textbf{Unit} & \textbf{Price} & \textbf{Amount}\tabularnewline
%% foreach invoice_items
%% (info) & (description) & (quantity) & (units) & (price) & (amount)\tabularnewline
%% end
\cline{1-6} 
 &  & \multicolumn{3}{l}{\textbf{Subtotal}} & (amount)\tabularnewline
\cline{3-6} 
%% foreach taxes
%% & & \multicolumn{3}{l}{(name_tax)} & (amount)\tabularnewline
%% end
%% if amount\_paid
%% \cline{3-6} 
%% &  & \multicolumn{3}{l}{Amount Paid} & (amount\_paid)\tabularnewline
%% end
\cline{3-6} 
 &  & \multicolumn{3}{l}{Balance Owing ((amount\_currency))} & (amount\_total)\tabularnewline
\end{tabular}

\noindent \vspace{20mm}


%% if company\_payment\_details
%% \noindent \begin{flushleft}
%% \textbf{PAYMENT METHODS}
%% 
%% \vspace{5mm}
%% 
%% (company\_payment\_details)
%% \end{flushleft}
%% end


\end{document}
