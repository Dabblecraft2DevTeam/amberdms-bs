%%
%% GENERIC ENGLISH INVOICE
%%
\documentclass[english]{article}
\usepackage[T1]{fontenc}
\usepackage[utf8]{inputenc}
\usepackage{geometry}
\geometry{verbose,letterpaper,tmargin=1cm,bmargin=1cm,lmargin=1cm,rmargin=1cm}
\usepackage{array}
\usepackage{graphicx}
\usepackage{setspace}
\usepackage{longtable}
\fontfamily{cmss}\fontsize{10pt}{12pt}\selectfont
\renewcommand{\familydefault}{\sfdefault}
\usepackage{helvet}
\makeatletter


%%%%%%%%%%%%%%%%%%%%%%%%%%%%%% LyX specific LaTeX commands.
%% Because html converters don't know tabularnewline
\providecommand{\tabularnewline}{\\}

\makeatother

\usepackage{babel}

\begin{document}
\noindent \includegraphics{(company_logo)}
		

\noindent \begin{tabular}{>{\centering}p{0.5\columnwidth}>{\centering}p{0.5\columnwidth}}
\noindent \begin{flushleft}

(company\_name)

(company\_address1\_street)

(company\_address1\_zipcode) (company\_address1\_city)

(company\_address1\_country)

\vspace{5mm}

{Email} (company\_contact\_email)

{Telefon} (company\_contact\_phone)

%% if company\_contact\_fax
%% {Fax} (company\_contact\_fax)
%% end

\par\end{flushleft}
& \begin{flushleft}

(customer\_contact)

(customer\_name)

(customer\_address1\_street)

(customer\_address1\_zipcode) (customer\_address1\_city)

(customer\_address1\_country)

\par\end{flushleft}
\tabularnewline
\end{tabular}

\noindent \vspace{10mm}

\noindent \begin{tabular}{ll}

\noindent \textbf{Rechnung}

\vspace{5mm}

%% foreach taxes
%% (name_tax) & (taxnumber) \tabularnewline
%% end
\end{tabular}

\noindent \vspace{10mm}


\noindent \begin{tabular}{ll}
\textbf{Konto-ID} & (code\_customer) \tabularnewline
\textbf{Rechnungsnummer} & (code\_invoice) \tabularnewline
\textbf{Rechnungsdatum} & (date\_trans) \tabularnewline
\textbf{F\"allig am} & (date\_due) \tabularnewline
\end{tabular}

\noindent \vspace{10mm}

\noindent \begin{tabular}{>{\raggedright}p{0.2\columnwidth}>{\raggedright}p{0.3\columnwidth}>{\raggedright}p{0.1\columnwidth}>{\raggedright}p{0.1\columnwidth}>{\raggedright}p{0.1\columnwidth}>{\raggedright}p{0.1\columnwidth}}
\textbf{Position} & \textbf{Beschreibung} & \textbf{Menge} & \textbf{Einheit} & \textbf{Preis} & \textbf{Betrag}\tabularnewline
%% foreach invoice_items
%% (info) & (description) & (quantity) & (units) & (price) & (amount)\tabularnewline
%% end
\cline{1-6}
\tabularnewline
 &  & \multicolumn{3}{l}{Zwischensumme} & (amount)\tabularnewline
\cline{3-6} 
%% foreach taxes
%% \tabularnewline
%% & & \multicolumn{3}{l}{(name_tax)} & (amount)\tabularnewline
%% end
%% if amount\_paid
%% \cline{3-6} 
%% \tabularnewline
%% &  & \multicolumn{3}{l}{Bezahlt} & (amount\_paid)\tabularnewline
%% end
\cline{3-6}
\tabularnewline
 & & \multicolumn{3}{l}{\textbf{Gesamttotal (amount\_currency)}} & {\bf(amount\_total)}\tabularnewline
\end{tabular}

\noindent \vspace{20mm}


%% if company\_payment\_details
%% \noindent \begin{flushleft}
%% \textbf{Zahlungsart}
%% 
%% \vspace{5mm}
%% 
%% (company\_payment\_details)
%% \end{flushleft}
%% end

\end{document}
