%%
%% GENERIC ENGLISH BALANCE SHEET REPORT
%%
\documentclass[english]{article}
\usepackage[T1]{fontenc}
\usepackage[utf8]{inputenc}
\usepackage{geometry}
\geometry{verbose,letterpaper,tmargin=1cm,bmargin=1cm,lmargin=1cm,rmargin=1cm}
\usepackage{array}
\usepackage{graphicx}
\usepackage{setspace}
\usepackage{longtable}
\usepackage{eurosym}
\fontfamily{cmss}\fontsize{10pt}{12pt}\selectfont
\makeatletter


%%%%%%%%%%%%%%%%%%%%%%%%%%%%%% LyX specific LaTeX commands.
%% Because html converters don't know tabularnewline
\providecommand{\tabularnewline}{\\}


\makeatother

\usepackage{babel}

\begin{document}
\noindent \includegraphics{(company_logo)}

\noindent \vspace{10mm}



\noindent \textbf{BALANCE SHEET}

\noindent \vspace{10mm}


\noindent \begin{tabular}{ll}
\textbf{As At} & (date\_end) \tabularnewline
\textbf{Date Created} & (date\_created) \tabularnewline
\end{tabular}

\noindent \vspace{10mm}

\noindent \textbf{Assets}

\noindent \begin{tabular}{>{\raggedright}p{0.8\columnwidth}>{\raggedright}p{0.2\columnwidth}}
%% foreach table_assets
%% (name_chart) & (amount)\tabularnewline
%% end
\cline{2-2} 
\textbf{Total Assets} & (amount\_total\_assets)\tabularnewline
\end{tabular}

\noindent \vspace{5mm}


\noindent \textbf{Liabilities}

\noindent \begin{tabular}{>{\raggedright}p{0.8\columnwidth}>{\raggedright}p{0.2\columnwidth}}
%% foreach table_liabilities
%% (name_chart) & (amount)\tabularnewline
%% end
\cline{2-2} 
\textbf{Total Liabilities} & (amount\_total\_liabilities)\tabularnewline
\end{tabular}

\noindent \vspace{5mm}


\noindent \textbf{Equity}

\noindent \begin{tabular}{>{\raggedright}p{0.8\columnwidth}>{\raggedright}p{0.2\columnwidth}}
%% foreach table_equity
%% (name_chart) & (amount)\tabularnewline
%% end
Current Earnings & (amount\_total\_current\_earnings)\tabularnewline
\cline{2-2} 
\textbf{Total Equity} & (amount\_total\_equity)\tabularnewline
\end{tabular}

\noindent \vspace{5mm}


\noindent \begin{tabular}{>{\raggedright}p{0.8\columnwidth}>{\raggedright}p{0.2\columnwidth}}
\textbf{Total Liabilities + Equity} & (amount\_total\_liabilities\_and\_equity)\tabularnewline
\end{tabular}




\end{document}
